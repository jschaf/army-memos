\documentclass[$if(fontsize)$$fontsize$,$endif$$if(lang)$$lang$,$endif$$if(papersize)$$papersize$,$endif$$for(classoption)$$classoption$$sep$,$endfor$]{$documentclass$}

\usepackage{setspace}

\usepackage{fixltx2e}% provides \textsubscript

\usepackage{fontspec}
\defaultfontfeatures{Ligatures=TeX,Scale=MatchLowercase}


% Times for the body.  Everyone expects Times for the body copy.
% However, AR 25-50 says to use Arial.
\setmainfont{TeX Gyre Termes}

% Helvetica for the letterhead.  AR 25-50 says use arial, AR 25-40
% says Helvetica.
\setsansfont{TeX Gyre Heros}

\usepackage[margin=1in,headsep=12pt,headheight=2\baselineskip]{geometry}

\usepackage{longtable,booktabs,dcolumn}

% Convert between units
\usepackage{xparse}
\ExplSyntaxOn
\NewDocumentCommand{\convertto}{mm}
   % #1 = em or ex (or any other unit)
   % #2 = dimen to convert
 {
  \texttt{#2~=~\fp_to_decimal:n { #2/1#1 }#1}
 }
\ExplSyntaxOff

\usepackage{graphicx}
\makeatletter
\def\maxwidth{\ifdim\Gin@nat@width>\linewidth\linewidth\else\Gin@nat@width\fi}
\def\maxheight{\ifdim\Gin@nat@height>\textheight\textheight\else\Gin@nat@height\fi}
\makeatother
% Scale images if necessary, so that they will not overflow the page
% margins by default, and it is still possible to overwrite the defaults
% using explicit options in \includegraphics[width, height, ...]{}
\setkeys{Gin}{width=\maxwidth,height=\maxheight,keepaspectratio}

\usepackage[unicode=true]{hyperref}
\hypersetup{breaklinks=true,
            bookmarks=true,
            pdfauthor={$author-meta$},
            pdftitle={$title-meta$},
            colorlinks=true,
            citecolor=$if(citecolor)$$citecolor$$else$blue$endif$,
            urlcolor=$if(urlcolor)$$urlcolor$$else$blue$endif$,
            linkcolor=$if(linkcolor)$$linkcolor$$else$magenta$endif$,
            pdfborder={0 0 0}}
\urlstyle{same} % don't use monospace font for urls
$if(links-as-notes)$
% Make links footnotes instead of hotlinks:
\renewcommand{\href}[2]{#2\footnote{\url{#1}}}
$endif$
$if(strikeout)$
\usepackage[normalem]{ulem}
% avoid problems with \sout in headers with hyperref:
\pdfstringdefDisableCommands{\renewcommand{\sout}{}}
$endif$
\setlength{\parindent}{0pt}
\setlength{\parskip}{6pt plus 2pt minus 1pt}
\setlength{\emergencystretch}{3em} % prevent overfull lines
$if(numbersections)$
\setcounter{secnumdepth}{5}
$else$
\setcounter{secnumdepth}{0}
$endif$
$if(verbatim-in-note)$
\VerbatimFootnotes% allows verbatim text in footnotes
$endif$
$if(lang)$
\ifxetex
  \usepackage{polyglossia}
  \setmainlanguage{$mainlang$}
\else
  \usepackage[$lang$]{babel}
\fi
$endif$

$if(title)$
\title{$title$}
$endif$
$if(subtitle)$
\subtitle{$subtitle$}
$endif$
$if(author)$
\author{$for(author)$$author$$sep$ \and $endfor$}
$endif$
$if(date)$
\date{$date$}
$endif$
$for(header-includes)$
$header-includes$
$endfor$

% ----------------
% Custom additions


\newlength{\topofletterhead}
\setlength{\topofletterhead}{.625in}

\newlength{\topofdodseal}
\setlength{\topofdodseal}{.5in}

\newlength{\lengthofdodseal}
\setlength{\lengthofdodseal}{1in}

% Customize lists to meet army regulations.
% 1st level: 1. 2. 3. flush with left margin
% 2nd level: a. b. c. indent 0.25 in
% 3rd level: (1) (2) (3) indent 0.5 in
% 4th level: (a) (b) (c) indent 0.5 in, same as 3rd level
\usepackage{enumitem}
\setlist[enumerate]{wide, labelwidth=\parindent}
\setlist[enumerate,1]{wide=0in, label=\arabic*.}
\setlist[enumerate,2]{wide=.25in, label=\alph*.}
\setlist[enumerate,3]{wide=.5in, label=(\arabic*)}
\setlist[enumerate,4]{wide=.5in, label=(\alph*)}

\renewenvironment{itemize}{\begin{enumerate}}{%
    \end{enumerate}\ignorespacesafterend% as suggested above
}

\usepackage{calc}

% Grid lines across whole page
% \usepackage{tikz}
% \newcommand{\Gridlines}{
%   \begin{tikzpicture}[remember picture, overlay]
%     \node[anchor=south west,inner sep=0pt] at (current page.south west)
%     {\tikz \draw[line width=0.2pt,color=gray!10,step=0.125in] (0,0) grid ++(8.5in,11in);};
%     \node[anchor=south west,inner sep=0pt] at (current page.south west)
%     {\tikz \draw[line width=0.3pt,color=gray!20,step=0.25in] (0,0) grid ++(8.5in,11in);};
%     \node[anchor=south west,inner sep=0pt] at (current page.south west)
%     {\tikz \draw[line width=0.4pt,color=gray!30,step=1in] (0,0) grid ++(8.5in,11in);};
%   \end{tikzpicture}
% }
% arbitrary fontscaling with \scalefont
\usepackage{scalefnt}


\usepackage{fancyhdr}

% textblock works in multiples of TPHorizModule and TPVertModule.  We
% need this to absolutely postion the letterhead and DOD seal.  Trying
% to place them by adjusting margins is fucking impossible.
\usepackage[absolute,overlay]{textpos}
\setlength{\TPHorizModule}{.25in}
\setlength{\TPVertModule}{.25in}

% Positioning the letterhead.
%
% AR 25-40, Appendix G-3, figure G-1 (text on the side of the figure):
%
% "Heading is centered on the page 5/8 inch from the top trim. All
% type is Helvetica bold. 'Department of the Army' is 10pt, all other
% type is 8 point except 'reply to', 'attention of', which is 6
% point. (Use of the phrase 'reply to, attention of' is optional.)
% The DOD seal is 1 inch in diameter"
%
% AR 25-30, 7-7.c covers the format of the letterhead: "Include the
% complete street address and ZIP+4 Code."  But then it says refer to
% AR 25-40, figure G-1 and G-2.
\newcommand{\daletterhead}[1]{%
  \fontsize{11pt}{11pt}\selectfont%
  \textbf{\textsc{#1}}\\%
}

\newcommand{\letterhead}[1]{%
  \fontsize{9pt}{9pt}\selectfont%
  \textbf{\textsc{#1}}%
}

\newcommand{\Letterhead}{
  \vbox{
    \sffamily
    \daletterhead{Department of the Army}
    $for(letterhead)$
    \letterhead{$letterhead$}$sep$\\
    $endfor$
  }
}

\newcommand{\memotype}[1]{\noindent\textsc{#1}}

\newcommand{\Signature}[1]{\noindent #1\\}

% No line in header
\renewcommand{\headrulewidth}{0pt}

% first page header
\fancypagestyle{firststyle} {
  \fancyhf{}
  \fancyhead[C]{%
    \begin{textblock}{12}(11,2.5)
      \Letterhead
    \end{textblock}
  }
}

% every other page header
\pagestyle{fancy}
\lhead{\scriptsize\OfficeSymbol\\ \footnotesize \textsc{Subject}: \Subject}
\fancyfoot[C]{\thepage}

% Official DOD seals: http://www.defense.gov/multimedia/web_graphics/
% You can convert eps to pdf with the 'epstopdf' tool available in
% tex-live.  Example shell commands:
%
% tlmgr install epstopdf
% epstopdf old_file.eps
%
% Position DOD seal absolutely, AR 25-40, Appendix G-3. Letterhead and
% memorandum stationery.  1/2in from the top edge of the page and 1/2in
% from the left edge of the page
%
% Regarding other logos, AR 25-30, 7-7.a states, "Do not print any
% seals, emblems, decorative devices, distinguishing insignia,
% slogans, office symbols, or mottos on letterhead or memorandum
% stationery except those approved or directed by HQDA"
\newcommand{\DODSeal}{%
\begin{textblock}{2}(2,2)
  \includegraphics[width=\lengthofdodseal]{images/dod_seal_black}
\end{textblock}
}

$if(officesymbol)$
\newcommand{\OfficeSymbol}{{\scshape $officesymbol$}}
$else$
\newcommand{\OfficeSymbol}{}
$endif$

\newcommand{\Subject}{$if(subject)$$subject$$else$Specify Subject$endif$}

% The document title
\makeatletter
\def\@maketitle{%
  {\small\OfficeSymbol} \hfill \@date \vskip 24pt%
  \memotype{Memorandum For Record} \vskip 12pt%
  \memotype{Subject}: \Subject \vskip 24pt%
  }
\makeatother


%%%%%%%%%%%%%%%%%%%%%%%%%%%%%%%%%%%%%%%%%%%%%%%%%%%%%%%%%%%%%%%%%%%%%%%%%%%%%%%%
\begin{document}
% Suppress overfull hbox warning for the DOD Seal
{\hfuzz=40pt \DODSeal}

% This would be much cleaner with a working \newgeometry, but that
% doesn't work unless you insert on the target page.  I tried
% afterpage, but that has issues with global scope.
%
% Since adjusting margins on separate pages is impossible, we fake the
% header height for the first page.  We need to see if the letterhead
% or the DOD seal is larger and \vspace that distance before inserting
% the title.

% The length from the top of the page to the bottom of the DOD seal.
\newlength{\bottomofdodseal}
\setlength{\bottomofdodseal}{\lengthofdodseal + \topofdodseal}

\newlength{\bottomofletterhead}
% suppress underfull errors from this fake insert of the letterhead
\settoheight{\bottomofletterhead}{\hbadness=\maxdimen \Letterhead}
\setlength{\bottomofletterhead}{\bottomofletterhead + \topofletterhead}
\setlength{\bottomofletterhead}{\maxof{\bottomofletterhead}{\bottomofdodseal}}

\newlength{\fakeheadersep}
\setlength{\fakeheadersep}{\bottomofletterhead - 1in}

\vspace*{\fakeheadersep}


$if(subject)$
{
 % Don't let maketitle force a new page.
  \let\newpage\relax\maketitle
  \maketitle
}
$endif$

$for(include-before)$
$include-before$

$endfor$
\thispagestyle{firststyle}

% AR 25-50 2-3.(2).c - Margins: "Do not right justify margins."
\raggedright

$body$

$for(include-after)$
$include-after$

$endfor$

% TODO: prevent signature from being an orphan, two things to try:
% stretch the bottom and top margin, wrap the signature in no-orphan
\vspace{48pt}

% Enclosures are governed by AR25-50 Section 4.2.  The general rule is
% consistency.  Enclosures are only listed if they are not mentioned
% in the body.

% Section 4.2.C.1 "For memorandums, capitalize the first letter in the
% first word of a listed enclosure; see figure 4–1 for an example. For
% letters, do not list the enclosures."

% Section 4.2.C.2 "Account for enclosures not identified in the body of the
% correspondence by indicating the total number. List each enclosure
% by number when you have two or more and describe each briefly (see
% fig 4–2)."

% Section 4.2.C.3 "When you have only one enclosure and it is not identified in the
% body, account for it without a number (see fig 4–3)."

% Section 4.2.C.4 "If identifying enclosures in the body, account for
% the enclosures without a number preceding “Encl/Encls.” When only
% some of the enclosures have been identified in the body, use “as
% stated” (as) as noted in paragraph 4–2c(6)). A list is not required
% (see fig 4–4)."
\noindent
\begin{minipage}[t]{.5\linewidth-3em}
$if(enclosures)$
\directlua{
  num_encls = 0
  $for(enclosures)$
  num_encls = num_encls + 1
  $endfor$
  if (num_encls == 1) then
    tex.print("Encl\string\\\string\\")
    tex.sprint("$enclosures$")
  else
    tex.sprint(num_encls .. " Encls")
    tex.print("\string\\begin{itemize}[leftmargin=*,noitemsep,topsep=0pt]")
    tex.print("$for(enclosures)$ \string\\item $enclosures$ $endfor$")
    tex.print("\string\\end{itemize}")
  end
}
$else$
\hspace{3em}
$endif$
\end{minipage}%
\hspace{3em}%
\noindent
\begin{minipage}[t]{.5\linewidth}
  $for(signature)$
  \Signature{$signature$}
  $endfor$
\end{minipage}
\end{document}
